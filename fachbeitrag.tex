\documentclass[a4paper, 12pt]{article}
\usepackage[a4paper,left=2.5cm,right=3cm,top=2cm,bottom=2cm]{geometry}
\usepackage[ngerman]{babel}
\usepackage[utf8]{inputenc}
\usepackage[]{csquotes}
\usepackage[style=authoryear-ibid,backend=biber]{biblatex}
\usepackage{mathptmx}
\usepackage[T1]{fontenc}
\usepackage{relsize}
\usepackage{graphicx}
\usepackage[]{blindtext}
\graphicspath{ {images/} }

\addbibresource{sources.bib}

\author{Lars Eppinger, s178121}
\title{Quantifizierung von Code-Qualität mithilfe statischer Code-Analyse}
\date{1. Juni 2020}

\begin{document}
\begin{titlepage}
    \begin{center}
        \vspace*{1cm}
        
        \Huge
        \textbf{Quantifizierung von Code-Qualität mithilfe statischer Code-Analyse}
        
        \vspace{0.5cm}
        \LARGE
        Sind Tools zur statischen Code-Analyse ein akkurater Weg zur Quantifizierung von Code-Qualität in objektorientierten Umgebungen?
        
        \vspace{1.5cm}
        
        \textbf{Lars Eppinger}

        \Large
        178121\\
        DAI-17
        
        \vfill
        
        Fachbeitrag zur Modulprüfung\\Wissenschaftlich angeleitete Berufspraxis
        
        \vspace{0.8cm}
        
        \includegraphics[width=0.4\textwidth{}]{university.png}
        
        \Large
        Hochschule für Telekommunikation Leipzig\\
        1. Juni 2020
        
    \end{center}
\end{titlepage}
\tableofcontents
\newpage

\section{Einführung}
Code-Qualität ist einer der entscheidenden Faktoren über Sicherheit, Wartbarkeit und Maintainability von Softwareprojekten. 
Durch neue Ansätze in der Softwareentwicklung, speziell der agilen Softwareentwicklung, ist es immer wichtiger, hohe Qualität in einer Codebasis sicherzustellen, um die Umsetzung neuer Anforderungen so schnell und fehlerfrei wie möglich zu gestalten.
Immer häufiger kommen für diesen Zweck Tools zur statischen Code-Analyse zum Einsatz.
In dieser Arbeit wird die Frage geklärt, ob diese Tools geeignet sind, um die Qualität einer Codebasis in objektorientierten Umgebungen zu quantifizieren.

\section{Code-Qualität}
\subsection{Definition}
\blindtext

\subsection{Merkmale}
\blindtext

\section{Statische Code-Analyse}
\subsection{Stand der Technik}
\blindtext

\section{Problematik}
\blindtext

bla \parencite{natural-language-processing}
\newpage
\printbibliography
\end{document}